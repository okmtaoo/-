\documentclass[a4j,12pt]{jsarticle}
\usepackage{semi}
\begin{document}
%目次
\pagenumbering{roman}
\setcounter{page}{1} % ページ番号1
\setcounter{tocdepth}{3}
\tableofcontents
\clearpage

%本文
\newpage
\pagenumbering{arabic}
\setcounter{page}{1} 
%緒言
\section{緒言}
近年e-Learningの普及とともに様々なシミュレータ教材の需要が増加している.さらに,パソコンだけでなく,タブレットやスマートフォンといった様々な端末での利用が見込まれ,その利用は学校だけでなく,塾や家庭など幅広い場での活躍が考えられる.

しかし,スマートフォンの性能はPCと比較して決して処理速度が速いとは言えず,一部の塾や通信講座では独自のハードウェアやアプリケーションの開発を行っていることから,従来のシミュレータ教材をそのまま適用するのは容易ではないのが現状である.したがって,従来のシミュレータ教材の処理速度を改善する必要がある.しかし,全てのシミュレータ教材を新規で開発した場合その労力は計り知れない.そのため,ただ処理速度を向上させるだけでなく,いかに変更点を少なく.処理速度を上げるのかが重要となってくる.
現在処理速度を向上させる手法として主にシェーダやWorkerを用いる方法がある.

しかし,シェーダとWorkerは共に問題点がある.シェーダは処理速度が大幅に向上するが,記述内容が複雑,増加するため手軽に実装することができないといった問題点がある.WorkerはJSをマルチスレッド化させるAPIである.しかし,描画処理を行うことができないためworkerで描画に必要なデータを計算で解き,別のファイルで描画を行う必要がある.そのため従来のソースコードを処理と描画で分ける必要がある.それらのデータの送受信を実装する必要があり,決して手軽に実装できないという問題点がある.

本研究室では主に音響のシミュレータ教材を制作し配布を行っていることを受け,音響のシミュレータで広く普及しているFDTD法を用いたシミュレータ教材を利用する.


本研究ではFDTD法により制作されたシミュレータ教材を手軽に処理速度を向上させる手法を考じ,制作,有用性を示すことを目的とする.

\newpage
%e-Learning
\section{e-Learning}
e-Learningとはelectronic Learningの略称である.その名の通りコンピュータやネットワークなどの情報技術を扱う学習形態である.e-Learningは1999年11月,アメリカフロリダ州にて開催されたTechLearn1999において初めて使用された.現在CAIやオンラインラーニングなど様々な学習形態が存在する. 

\subsection{e-Learningの定義}
米国の組織学習・人材開発に関する世界最大の会員制組織であるASTD(American Society for Training & Development)によると「e-Learningとは,明確な学習目的のために,エレクトロニクス技術によって提供,可能とされた伝達されるあらゆるものである.」と,定義されている.

一般的に情報技術を用いているもの全てをe-Learningと「広義」でとらえる場合と,非同期型オンライン方式を想定した「狭義」である場合とさまざまであり、一概に定義通りではない.テクノロジを活用した学習形態の幅が広がる現在の傾向では「情報技術によるコミュニケーション・ネットワークなどを活用した主体的な学習」を総称してe-Learningと定義するのが一般的である.

\subsection{e-Learningに含まれる学習形態}
e-LearningはIT技術を用いた様々な学習形態を総称して呼ぶ.したがって一言でe-Learningと呼んでも複数の種類が存在する.そこで本章ではe-Learningの種類と特徴・開発の経緯などを紹介する.

\subsubsection{CAI}
CAI(Computer-Assisted Instruction)は1950年代後半,米国で兵員教育の目的として発足.1970年代から80年代にかけて,「講師と受講者が長時間同じ場所に居なければならないという問題を,コンピュータを用いることにより軽減できないだろうか」という,パソコンを教育に活用する学習形態として注目された.CAIはスタンドアローン環境が前提であり,受講者はマニュアル通りの単純な,決められた学習を行う形式だった.そのため受講者のレベルに応じた教育,柔軟なカリキュラムの提示を行えず,効率的な学習を十分に行えなかった.また,ネットワークの利用には至らなかったため,受講者の進捗の確認などは人間が行う必要があった.

1980年代に入るとハードウェアの開発が進み,従来の大型コンピュータで実行されていたCAIがパーソナルコンピュータでも実行可能となり,モデムやスキャナーなどの周辺機器の開発によりメディアとしての能力が向上した.

1990年代にはコンピュータネットワークが発達,個々人の情報リテラシー能力の向上を受け,CBTやWTBが考案,2000年代以降には,これらの概念を活用したe-Learningが盛んに用いられるようになった.

\subsubsection{CBT(Conputer-Based Training)}
CBT(Conputer-Based Training)は,1986年に教育,調査,測定分野の非営利団体であるETS(Esucational Testing Service)が,大学生の能力別クラスの編成用のテストとして利用したことから始まった.当時はットワークが普及していなかったため,CD-ROMの大容量な特性を活かし,動画や音声などを利用したインタラクティブなコンテンツが効果的に利用された.また,受講者管理やコンテンツ配信を行うサーバを必要としないため,比較的容易に導入することが可能だった.しかし.CD-ROMを制作するためのコストと,配布後の修正が難しく,各個人の進捗状況を一括して管理することが困難であった.

その後,1990年代に入り,IT系企業の認定資格試験に利用されたことから大きく発展し,現在では公的要素の強い試験など様々な分野でCBT化が進んでいる.日本国内においても国家試験の一つである情報処理技術者試験で2003年から導入されているように,現在でも学校教育や企業内教育の現場で,幅広く取り入れられている.

\subsubsection{WBT}
1990年代中盤から後半にかけて,ソフトウェアの進化とネットワークの普及に伴い,インターネットやイントラネットなどのネットワークを通して教育コンテンツ学習者に提供するWBT(Web-Based Training)という学習形式である.従来のWBTとCBTとの大きな違いは,ネットワークを介していることである.ネットワークを用いることによる恩恵は以下の3つが挙げられる.

\begin{enumerate}
\item 知識の変化に合わせて教育内容の更新が容易なこと
\item 双方向通信が可能になり,受講者と講師,もしくは受講者同士間のインタラクティブなコミュニケーション,及び受講者の進捗状況の把握が可能になったこと
\item Webは世界標準のプロトコルが存在するため,教材互換性,汎用性,操作性が向上したこと
\end{enumerate}

\subsubsection{EPSS(Electronic Performance Support System)}
EPSS(Electronic Performance Support System)とは,IT技術を活用して,業務中に必要な知識やツールの提供を行うことで,パフォーマンスの向上を支援するシステムのことである.

例としてコールセンターやビデオマニュアルが挙げられる.インタラクティブ学習やパフォーマンス・サポート・システムの分野を専門にしているグロリア・ゲーリー氏によれば「他社による最小限のサポートと介入で,ジュブパフォーマンスが生まれることを可能にするモニタリングシステム,情報,ソフトウェア,支援,データ,画像,ツール,評価の全ての内容に働く人々が簡単にアクセスできる.また,それらを個々人が手元で見ることができ,直ちに必要な内容を入手できる仕組みを持つ,使いやすく統合された電子的環境」と定義している.

CBTやWBTといった,業務から一度離れて学習する形態と比較し,より業務遂行に直結した知識を会得できるシステムであることが特徴であり,近年はより効率化を図るために,これまで活用していた紙のマニュアルを電子化する動きがある.米国においてはユーザのパフォーマンスに合わせた設計を行うことをPCD(Performance Centered Design)と呼び,EPSSと同義として扱われる場合がある.
\subsubsection{Knowledge Management System}
Knowledge Managementとは,個人の持っている知識,情報を組織で共有し,コミュニティ全体の創造性を向上させるための手法である.個人の持つデータだけでなく,経験やノウハウを含んだ幅広い物を指す.

Knowledge Managementとe-Learningは独立したものと扱われることが多かった.しかし,米国でe-Learningの分野においても,マニュアルの理解や受講者間での知識の共有や生成の場を設けることの重要性を訴えられる様になった.そのことが実現した場合e-LearningとKnowledge Managementの区別がなくなるため,両者を融合し,e-Learningの一環として捉えるようになった.

\subsubsection{Blended Learning}
Blendes Learningは,e-Learningと集合研修を組み合わせた形態ことである.

それらの組み合わせは,他者からの刺激を受けることで学習意欲の向上,相互作用をによって理解促進,知識の整理が行えるという利点がある.
\begin{description} 

\item[e-Learnig + 集合研修]\mbox{}\\ 
事前学習を済ませた後に,教室でインタラクティブな学習を行う

\item[e-Learnig + 集合研修]\mbox{}\\ 
事前学習と教室研修を行い,その後にバーチャルクラスで学習を行う

\item[集合研修 + e-Learnig]\mbox{}\\ 
集合研修後にフォローアップのためにセルフ学習を行う

ディスカッションバーチャルクラス,チュータからのメンタリングを含む
\end{description}
このように様々な形態が存在するが,e-Learningに適した内容化を見極める必要である.

\subsection{e-Learningの効果}
e-Learningには様々な形態が存在するが,学習形態ごとに違いがあるが,e-Learnigという大きな枠組みで考えたときのメリットとデメリットをまとめる.
\subsubsection{e-Learningのメリット}
\begin{large}受講者のメリット\end{large}
\begin{enumerate}
\item 場所と時間に制限がない
\item 受講者のレベルや習熟度に合わせたコンテンツの提供できる
\item 研修終了後に受講者間や講師との間でインタラクティブなコミュニケーションが可能
\item 世界標準のプロトコルを用いることで,教材の互換性,汎用性,及び操作性が向上
\item 予め筋道を立ててシステムを組むため,全体の筋道が決まっており,一括して見やすい
\item 場所の移動を省くことが可能なため,時間とコストを削減できる
\item 最新の情報を安価で早く入手し,学習することができる
\end{enumerate}

\begin{large}制作側のメリット\end{large}
\begin{enumerate}
\item 集団教育と比較すると安価に提供できる
\item マルチメディアを用いたインタラクティブな教材の制作が可能
\item 学習の進捗状況をリアルタイムで把握することができるため,管理が用意
\item 知識の変化に合わせ,最新の内容に更新した教材の把握が容易
\end{enumerate}

\subsubsection{e-Learningのデメリット}
\begin{description} 

\item[(1)コンピュータ,またはインターネット環境が整っているところでしか学習できない]\mbox{}\\ 
本などの資料は,持ち運びが容易なため,電車の中などで資料を持ち込むことや書き込むことで学習することが可能だが,e-Learningでは難しい.しかし,小型の端末などで学習を行えるソフトがリリースされており,以前と比べるとその欠点は薄れつつあるといえる.

\item[(2)就業時間中の学習を公認する必要がある]\mbox{}\\ 
集合研修は,会社がその時間の学習を公認しているが,書籍での学習は就業時間外として扱われることが多く,自主学習形態をとることが多い.しかし,e-Learningはインターネットが使えるコンピュータで学習することが一般的であり,セキュリティ上社外からのアクセスは禁じられていることが多く,学習するには社内で行う必要がある.その場合就業中か,仕事を終えて自分の席で学習することになる.自分の席で学習する場合,他者の目にはばかられ,思うような学習は難しいと思われる.就業時間中の学習は多くの企業で公認されにくいため,教材をリリースしても誰も学習しないという状況も多い.

\item[(3)システム投資コストがかかる]\mbox{}\\ 
e-Learningを使用する環境を構築するには教材を制作するだけでなく,サーバ管理コストやコンテンツ維持コストや補助的サーバなどが挙げられる.また,レポート提出や質疑応答システムと統合した環境を整えるとそれなりに費用がかかってします.システムの投資コストが従来の教育コストより高くなってしまう場合,e-Learningのメリットが無くなってしまう.

\item[(4)コニュニケーションや,体で覚える技術教育には向かない]\mbox{}\\
実際に対話を行う集合研修に対し,e-Learningは成約が多い.しかし,誰かが一方的に話すタイプの集合研修と比較すれば,この弱点は解消されたといえる.

\item[(5)シミュレータ搭載が困難]\mbox{}\\
 現在実用化されているe-Learning教材のなかで,文系科目に該当する大学などで多く実用化されている.その背景に容易に制作が可能なことが挙げられる.容易に作成できることからコンバータを用いることが多いが,コンバータはシミュレータやアニメーションに対応されておらず,実験などの補修学習が要の理系科目において十分な効果を期待することはできない.
\end{description}
\newpage
%シミュレータ教材
\section{シミュレータ教材}
\subsection{シミュレータ教材の定義}
シミュレータ教材とはシミュレーションを行う教材のことである.

現実世界の問題はシステムが複雑で予測が困難な場合が多く,実システムで試すにはコストや時間がかかるといった場面が多く存在する.
しかし,コンピュータ上に現実世界を再現することにより,実際のシステムの制作,変更することなく様々な条件下の環境を再現,システムの挙動を調べることができる.このコンピュータ上に現実世界を仮想的に制作するシステムをシミュレータと呼ぶ.

したがって,シミュレータ教材とは様々な条件下の現象を実際に生成することなく再現することで,対象を視認,理解させるための教材である.

一般的には,航空シミュレータや音響シミュレータなど,シミュレーションの結果を体験させるためのシステムを指すことが多いが,本研究では,コンピュータ上で動作し,ブラウザ上でパラメータを変更させ,その結果を同じブラウザ上で確認できるソフトウェアのことを指す.
\subsection{FDTD法}
音場のシミュレーションは大きく分けると定常解析手法と非定常解析手法の2つに分類される.定常解析手法とは対象が定常状態に達したときの音場や位相を計算する手法である.音波は時間とともに変動するが定常状態のため時間に依存することなく空間分布が求められる.

一方,非定常解析手法は時間によって変化する場を取り扱う.したがって音圧などの物理量が時間と共に変化する場を扱うため,波形そのものの計算が得意である.

本章では本研究に使用し,音響のシミュレータにおいて度々用いられる非定常手法の一つであるFDTD方について論じていく.

\subsubsection{FDTD法の概要}
FTDT(Finite Difference Time Domain)法は,1966年にK.S.Yeeによって提案された.K.S.Yeeはマクスウェル方程式が電解と磁界の連立方程式であることに着目し,中心差分に電磁界の時間及び空間配置を考案した.後に多くの科学者によって発展し,現在では電磁波の支配式を求めるだけでなく,流体力学や熱工学だけでなく波動光学にも用いられることが多くなった.

\subsubsection{計算方法}
物体に力を加えると物体は移動する,それと同様に空気も力が加わると動きが生じる.それはニュートンの第2方式,すなわち,運動方程式で記述される.空気には物体に動きが加わることによって生じる圧力と,運動に依存しない圧力が存在し,音響において前者を音圧,後者を大気圧と呼ぶ.音波の伝搬する空間を音場と呼ぶ.

ある空間においてx,yの直交座標の寸法がそれぞれ$\Delta x$,$\Delta y$とし,密度をρ,音圧をpとし,それぞれの変異を$u_x$,$u_y$とする.$\Delta x$が微小であると仮定したときのx方向の運動方程式は
\begin {eqnarray}
\label{eq:x}
\rho \frac {\partial t^2} {  \partial ^2 u_x } = - \frac { \partial p } { \partial x } \
\end{eqnarray}
y方向も同様に
\begin {eqnarray}
\label{eq:y}
\rho \frac {\partial t^2} {  \partial ^2 u_y } = - \frac { \partial p } { \partial y } \
\end{eqnarray}
と表される.体積弾性率を$k[N/m^2]$としたときの音圧に関する連続方程式は
\begin {eqnarray}
\label{eq:w}
p = -k ( \frac { \partial u _ x } { \partial x } + \frac { \partial u_y } { \partial y } \ )
\end{eqnarray}
となる.
式(\ref{eq:x}),(\ref{eq:y}),(\ref{eq:w})を物理現象を方程式化した式,すなわち,支配式として音場の解析を行う.これらの式を時間微分した式を以下に示す.

\begin {eqnarray}
\label{eq :  x-2}
\rho \frac{ \partial v_x } { \partial t } = - \frac { \partial p }{ \partial x }
\end{eqnarray}
\begin {eqnarray}
\label{eq :  y-2}
\rho \frac{ \partial v_y } { \partial t } = - \frac { \partial p }{ \partial y }
\end{eqnarray}
\begin {eqnarray}
\label{eq :  w-2}
\frac{ \partial p } { \partial t } = - k ( \frac { \partial v_x }{ \partial x } + \frac { \partial v_y }{ \partial y } ) 
\end{eqnarray}
となる.

音圧は時間と共に連続的に変化するが,

\subsection{シミュレータ教材のメリット}
シミュレータを使用することに得られるメリットを以下に示す.
\begin{enumerate}
\item コンピュータで疑似体験

シミュレータ教材とは,現実の物理現象をモデル化し,コンピュータ上で計算することで,模擬体験できる藻である.仮想環境の中での学習により,従来では考えられない程の学習効果を得ることができ,学習者の意欲向上に繋がると考えられる.
\item 等しく均一の条件で学習が可能

学習者自身が,実験の条件や変数を変えながら実験に参加することが可能である.実験変数の変更を安全に,何度でも変更することが可能なため,繰り返し学習が可能となり,学習者は全員均一の条件で学習できる.
\item 安全性

シミュレータ教材により学習者は,教室内で費用,時間のかかるものや.危険で簡単に実施できない実験を疑似体験することができる.
\end{enumerate}

\subsection{シミュレータ教材のデメリット}
シミュレータ教材は通常,全ての現象を思考要素とせず,対象要素を絞り込むことにより,要素が現象に与える影響を検証することが目的である.したがって,結果が不確定な事象を検証することは極めて困難である.

コンピュータを用いた積算によるシミュレートは,基本線形近似による計算のため,非線形を含む自然現象をシミュレートする場合は必ず誤差が生じる.良好な結果が得たければ,誤差見積もりが重要になる.

そして,シミュレータ教材を使うにはシミュレータ教材を作る必要がある.開発者はシミュレータを開発するためのソフトウェアを保持した状態で,プログラミングの知識や主題についての専門的な知識が要求されるため,開発者が限定される.
\newpage
%プロセッサ
\section{プロセッサ}
本章ではプロセッサ,CPUやGPUの歴史や相違点といった内容を解説する.
\subsection{CPU}
\subsubsection{CPUの概要}
\subsubsection{CPUの歴史}

\subsection{GPU}
GPUはGraphics Processing Unitの略称である.元々GPUは画像処理,大量の計算を行うことを目的として開発されたハードウェアである.しかし,近年はその処理能力の高さからディープラーニング等画像処理以外の分野での活躍も多く見られるようになった.
本章では.GPUの概要から歴史,処理の仕方,役割について述べる.

\subsubsection{GPUの概要}
GPUは元々画像処理を行うことを目的としている並列演算ハードウェアである.NVIDIAによってグラフィックス処理以外の演算を行わせるようにしたGPGPU(General-Purpose computing on GPUs)の開発環境CUDAが一般公開されると,GPGPUの研究が注目を浴び,現在ではその高い処理能力を活かしディープラーニングに広く普及している.

一般的にGPUとは,ビデオカード(グラフィックボード)と呼ばれるハードウェアとして販売されているハードウェアを指し,ビデオカードとLSIチップを厳密に区別されずにGPUと呼ばれることが多い.
\subsubsection{GPUの歴史}

\subsubsection{GPUの処理方法}
\subsubsection{GPUの活躍}

\subsection{CPUとGPUの相違点}
\newpage

\section{プログラム}
本章では本論文の核となるプログラミングについて解説を行う.
\subsection{JavaScript}
\subsection{WebWorker}
\subsection{OffscreenCanvas}
\newpage

\section{実装したシミュレータ}
\subsection{開発したシミュレータ}
\subsection{実験方法}
\newpage

\section{実験結果}
\newpage

\section{結言}
\newpage

\section{謝辞}
本研究の遂行及び本論文の制作に当たり,助言をくださった須田研究室の仲間に深く感謝の意を表します.そして,本論文の制作に当たり多大なるご指導及びご助言をいただきました須田宇宙准教授に深く感謝いたします.
\newpage

\section{参考文献}
\newpage

\section{付録 制作したプログラム}




\end{document}