% !TEX encoding = UTF-8 Unicode

\documentclass[a4j,12pt]{jsarticle}
\usepackage[dvipdfmx]{graphicx}
\usepackage{newtxtext} % アルファベットなどをTimesNewRomanで
%\usepackage{newpxtext} % アルファベットなどをPalatinoで
% 両方共コメントアウトするとComputer Modernで
\usepackage{listings}
\usepackage{okumacro}
\usepackage{graphicx}
\usepackage{url}
\usepackage{tabularx}
\usepackage{colortbl}
%\usepackage{float}
\usepackage{makeidx}
\usepackage{fouriernc}
\usepackage[deluxe]{otf}
\usepackage{mediabb}
\usepackage{amsmath}
\usepackage{moreverb}
%\usepackage{verbatim}

\usepackage{verbatim}
\usepackage{multirow}

% \usepackage[margin=2cm]{geometry}
\usepackage[top=20truemm,bottom=20truemm,left=30truemm,right=20truemm]{geometry}


\usepackage{dcolumn}
\usepackage{ascmac}
\usepackage{color}
\usepackage{eclbkbox}
\usepackage{itembkbx}
\usepackage{enumerate} %箇条書きを装飾する際に使う
\usepackage{multicol}
% 自分で追加
\usepackage{comment}%複数行コメントアウト
\usepackage{multirow}%表結合
\usepackage{here} %図や表などを強制的に出力
\usepackage{fancybox} %四角で囲む
\usepackage{boxedminipage} %箇条書きを四角で囲む
\usepackage{caption} %キャプションの設定
\usepackage{longtable} %表がページをまたがるときの処理
\usepackage{slashbox} %表に斜線を入れる
% \usepackage{emathT} %表に斜線を入れる
\usepackage{framed} %箇条書きを枠で囲む
\newcolumntype{I}{!{\vrule width 1.5pt}}% 縦線の一部を太くする(Iで指定する)
% 横線の一部を太くする(wlineで範囲を指定する)
\newlength\savedwidth
\newcommand{\wcline}[1]{\noalign{\global\savedwidth\arrayrulewidth\global\arrayrulewidth 1.5pt} \cline{#1}
\noalign{\global\arrayrulewidth\savedwidth}}
\usepackage{lscape} % 表を横向きに表示

% 数字を丸で囲む
\def\MARU#1{\leavevmode \setbox0\hbox{$\bigcirc$}%
\copy0\kern-\wd0 \hbox to\wd0{\hfil{#1}\hfil}}
%%%%%%%%%%%%%%%%%%%%%%%%%%%%%%%%%%


\makeatletter % プリアンブルで定義開始
\renewcommand{\presectionname}{第}
\renewcommand{\postsectionname}{章}
\renewcommand{\appendixname}{付録}


%章,節,項の文字サイズ
\def\section{\@startsection {section}{1}{\z@}{3.5ex plus -1ex minus -.2ex}{2.3 ex plus .2ex}{\Large\bf}}
\def\subsection{\@startsection {subsection}{1}{\z@}{3.5ex plus -1ex minus -.2ex}{2.3 ex plus .2ex}{\Large\bf}}
\def\subsubsection{\@startsection {subsubsection}{1}{\z@}{3.5ex plus -1ex minus -.2ex}{2.3 ex plus .2ex}{\large\bf}}


\usepackage{caption} %キャプション文字サイズ(大)
\captionsetup[figure]{font=normalsize}
\captionsetup[table]{font=normalsize}
\captionsetup[equation]{font=normalsize}

% 図の番号を"<章.節の番号> - <図の番号>" へ
\renewcommand{\thefigure}{\thesubsection-\arabic{figure}}
%\renewcommand{\thefigure}{\thesection-\arabic{figure}}
% 表の番号を"<章.節の番号> - <表の番号>" へ
\renewcommand{\thetable}{\thesubsection-\arabic{table}}
% 数式の番号を"<章.節の番号> - <数式の番号>" へ
\renewcommand{\theequation}{\thesubsection-\arabic{equation}}

% 章と節が進むごとに図の番号をリセットする
\@addtoreset{figure}{subsection}
% 章と節が進むごとに表の番号をリセットする
\@addtoreset{table}{subsection}
% 章と節が進むごとに数式の番号をリセットする
\@addtoreset{equation}{subsection}

% 章と節が進むごとに注釈の番号をリセットする
% \@addtoreset{footnote}{subsection}

% 注釈の設定
\renewcommand{\thefootnote}{\fnsymbol{footnote}}


\makeatother % プリアンブルで定義終了


%%%%%%% グラフックパス
\graphicspath{{fig/}}
 
\makeindex
\begin{document}

% フォントサイズ・行間
\fontsize{20pt}{15pt}\selectfont

% 表紙
\thispagestyle{empty} % ページ番号削除

\begin{center}
\huge
\vspace*{\stretch{2}}
平成29年度 卒業論文\\[50pt]
\HUGE
GPU利用による\\
シミュレータ教材の\\
演算速度\\
\huge
\vspace*{\stretch{6}}
指導教員 須田 宇宙 准教授\\[40pt]
千葉工業大学 情報ネットワーク学科\\[10pt]
須田研究室\\[60pt]
1432103 \hspace{50pt} 氏名 中嶋 大貴\\[75pt]
\end{center}

\begin{flushright} 
\huge
提出日 2018年1月29日
\vspace{\stretch{2}}
\end{flushright}

\newpage
\thispagestyle{empty} % ページ番号削除
%\input{input0}

% フォントサイズ・行間
\fontsize{10pt}{15pt}\selectfont

% 目次
\pagenumbering{roman}
\setcounter{page}{1} % ページ番号1
\setcounter{tocdepth}{3}

\newpage
\tableofcontents
%\pagestyle{empty}

%\thispagestyle{empty} 

\newpage
%\thispagestyle{empty}
\listoffigures

\newpage
%\thispagestyle{empty}
\listoftables




% 本文
\newpage
\pagenumbering{arabic}
\setcounter{page}{1} % ページ番号1
% 緒言
\input{input1}

\newpage
\input{input2}

\newpage
\input{input3}

\newpage
\input{input4}

\newpage
\input{input5}

\newpage
\input{input6}

\newpage
\input{input7}

\newpage
\input{input8}

\newpage
\input{input9}

\newpage
\input{input10}

% 結言
\newpage
%\input{input11}

% 謝辞
\newpage
\thispagestyle{empty} % ページ番号削除
%\input{input12}

%参考文献
\newpage
\thispagestyle{empty} % ページ番号削除
%\input{bibliography.tex}

%付録
\appendix
\section{作成したプログラム}

\section*{CPU.html}
\label{ CPU.html}
\listinginput{1}{CPU.html}
\newpage

\section*{manysource.js}
\label{manysource.js}
\listinginput{1}{manysource.js}
\newpage

\section*{GPU.html}
\label{GPU.html}
\listinginput{1}{GPU.html}
\newpage

\section*{manysourcegpu.js}
\label{manysource.js}
\listinginput{1}{manysource.js}
\newpage

\section*{index.html}
\label{index.html}
\listinginput{1}{index.html}
\newpage

\section*{study.js}
\label{study.js}
\listinginput{1}{study.js}
\newpage
\listinginput{1}{study.js}
\newpage

% \renewcommand{\thetable}{\Alph{section}.\arabic{figure}}
% \renewcommand{\thefigure}{\Alph{section}.\arabic{table}}

% \def\thesection{付録\Alph{section}}

%\newpage
% 辞書の作成方法
%\input{appendix1}

%\newpage
% Juliusのプログラム
%\input{appendix2}

\newpage
% 確認試験の内容
%\input{appendix3}

\newpage
% 質問紙の内容
%\input{appendix4}

\newpage
% 学習シナリオ
%\input{appendix5}

%\newpage
%\input{sakuin}

\printindex

\end{document}